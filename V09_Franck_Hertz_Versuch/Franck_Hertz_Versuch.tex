\documentclass{scrartcl}
\usepackage[latin1]{inputenc}
\usepackage[T1]{fontenc}
\usepackage[ngerman]{babel}
\usepackage{amsmath}
\usepackage{amssymb}
\usepackage{icomma}
\usepackage{nicefrac}
%\usepackage[dvips]{graphicx}
%\usepackage{floatflt}
%\usepackage{enumitem}
%\usepackage{babel}
\usepackage{blindtext}
%\usepackage{showframe}
\usepackage{calc}
\usepackage{wrapfig}
\def\BILD{\rule{0.4\textwidth}{4cm}}

\usepackage{graphicx}
\usepackage{placeins}
\usepackage{multirow}
\usepackage{subfig}
\usepackage{url}

\renewcommand{\topfraction}{.85}
\renewcommand{\bottomfraction}{.7}
\renewcommand{\textfraction}{.15}
\renewcommand{\floatpagefraction}{.66}
\renewcommand{\dbltopfraction}{.66}
\renewcommand{\dblfloatpagefraction}{.66}
\setcounter{topnumber}{9}
\setcounter{bottomnumber}{9}
\setcounter{totalnumber}{20}
\setcounter{dbltopnumber}{9}
\setlength{\intextsep}{0cm plus1cm minus1cm}
\setlength{\parindent}{0cm}
\pdfminorversion = 5
\usepackage{setspace}
\onehalfspacing

\begin{document}
\title{Versuch 9: Franck-Hertz-Versuch}

\date{\today}

\author{Gruppe 5a: Gia-Danh Lam, Nils Haldenwang}

\maketitle
\tableofcontents

\newpage
\section{Einleitung und theoretischer Hintergrund}
\label{sec:einleitung}

Der Franck-Hertz-Versuch wurde erstmals in den Jahren 1911 bis 1914
von James Franck und Gustav Ludwig Hertz durchgef�hrt und belegt die
Existenz von diskreten Energieniveaus in Atomen. Im Jahre 1925
erhielten sie daf�r den Nobelpreis.

Diskrete Energiezust�nde in Atomen unterscheiden sich durch die
Energiedifferenzen zwischen ihnen. Durch Zuf�hrung von exakt einer
solchen Energiedifferenz k�nnen Elektronen auf ein h�heres Niveau
gelangen, durch Abgabe der Energie in Form eines Photons ist es ihnen
ebenso m�glich wieder auf ein niedrigeres Niveau zu fallen.

Franck und Hertz machten sich ineleastische St��e von beschleunigten
Elektronen und Gasatomen zu nutze um die Quantisierung bei Aufnahme
und Abgabe von Energie zu zeigen. Eine schematische Darstellung der
R�hre und ihrer Beschaltung findet sich in Abbildung \ref{fig:}.

\begin{figure}[htb!]
  \centering
  \includegraphics[width=20cm]{pics/dummypic}
  \caption{Schematische Darstellung der R�hre und ihrer Beschaltung}
  \label{fig:}
\end{figure}

Die R�hre ist gef�llt mit Quecksilber. Die Elektronen werden mittels
Gl�helektrischem Effekt aus der Heizkathode $ K $ herausgel�st und
mittels der Beschleunigungsspannung $ U_{A} $ zur Gitteranode $ A $
beschleunigt. So sie hinreichend beschleunigt wurden, erreichen sie
die Gegenkathode $ GK $ gegen die Gegenspannung $U_{S}$. Der dabei
flie�ende Strom $I_{S}$ wird in Abh�ngigkeit von $U_{A}$ gemessen.

Neben den inelastischen St��en finden im Allgemeinen aber vor allem
elastische St��e statt. Die gr��ten Energie�bertragungen finden bei
zentral-elastischen St��en statt, dabei gen�gt die Energie�bertragung
$ \Delta W $ f�r ruhende Hg-Atome der folgenden Gleichung:

\begin{equation}
  \label{eq:1}
  \Delta W = \frac{4 m_{e} \cdot m_{Hg}}{(m_{e} + m_{Hg})^2} \cdot
  W_{El} \text{ mit } m_{Hg} = 3.7\cdot 10^5 m_{e}
\end{equation}

Aus Gleichung \eqref{eq:1} ist leicht ersichtlich, dass die Elektronen
kaum Energieverlust erleiden, lediglich ihre Bewegungsrichtung kehrt
sich um. Sie durchlaufen das Feld nun in entgegensetzter Richtung und
gelangen maximal zur�ck bis zur Kathode, wo schlie�lich wieder eine
erneute Beschleunigung in Richtung Anode beginnt.

Nicht zentrale St��e k�nnen in einer Ablenkungsrichtung senkrecht zum
Feld resultieren, sodass die Elektronen zur R�hrenwand gelangen und
dort Raumladungen bilden. Die Raumladungen sto�en weitere nachfolgende
Elektronen wieder ab.

Elastische St��e f�hren also h�chstens zu einer kontinuierlichen
Stromminderung durch abflie�ende Elektronen �ber die R�hrenwand. Dies
hat auf die Genauuigkeit der Messung keinen Einfluss, da lediglich der relative
Wert des Stroms von Bedeutung ist.

\section{Aufgaben}
\label{sec:aufgaben}

\subsection{Anregungsenergie und Wellen�nge}
\label{sec:anreg-und-well}

Mach ich morgen.

\subsection{Energiebetrachtung zum Sto�}
\label{sec:energ-zum-sto3}

Mach ich morgen.




\end{document}