\documentclass{scrartcl}
\usepackage[latin1]{inputenc}
\usepackage[T1]{fontenc}
\usepackage[ngerman]{babel}
\usepackage{amsmath}
\usepackage{amssymb}
\usepackage{icomma}
%\usepackage[dvips]{graphicx}
%\usepackage{floatflt}
%\usepackage{enumitem}
%\usepackage{babel}
\usepackage{blindtext}
%\usepackage{showframe}
\usepackage{calc}
\usepackage{wrapfig}
\def\BILD{\rule{0.4\textwidth}{4cm}}

\usepackage{graphicx}
\usepackage{placeins}
\usepackage{multirow}
\usepackage{subfig}
\usepackage{url}

\renewcommand{\topfraction}{.85}
\renewcommand{\bottomfraction}{.7}
\renewcommand{\textfraction}{.15}
\renewcommand{\floatpagefraction}{.66}
\renewcommand{\dbltopfraction}{.66}
\renewcommand{\dblfloatpagefraction}{.66}
\setcounter{topnumber}{9}
\setcounter{bottomnumber}{9}
\setcounter{totalnumber}{20}
\setcounter{dbltopnumber}{9}
\setlength{\intextsep}{0cm plus1cm minus1cm}
\pdfminorversion = 5
\usepackage{setspace}
\onehalfspacing

\begin{document}
\title{Versuch 11: Atomspektren}

\date{\today}

\author{Gruppe 5a: Gia-Danh Lam, Nils Haldenwang}

\maketitle
\tableofcontents

\newpage
\section{Einleitung und theoretischer Hintergrund}
\label{sec:einleitung}

In einer Atomh�lle gebundene Elektronen k�nnen sich auf
unterschiedlichen Energieniveaus aufhalten. Durch Absorption eines
Photons der Energie $ E = h \cdot \nu $ kann das Elektron auf ein
h�heres Energieniveau angehoben werden. Dabei entspricht die Energie
des Photons genau der Energiedifferenz der beiden Niveaus. Umgekehrt
wird ein Photon entsprechender Energie emittiert, wenn ein Elektron
von einem h�heren Energieniveau auf ein niedrigeres f�llt. Da die
Energieniveaus diskret sind, m�ssen auch ihre Differenzen diskret
sein. Dies ist die Ursache daf�r, dass Emissionspektren von Atomen
stets aus einzelnen Linien bestehen. Man bezeichnet sie daher auch als
Linienspektren.

\subsection{Spektrum des Wasserstoffatoms}
\label{sec:spektr-des-wass}




\end{document}



