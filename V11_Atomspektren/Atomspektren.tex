\documentclass{scrartcl}
\usepackage[latin1]{inputenc}
\usepackage[T1]{fontenc}
\usepackage[ngerman]{babel}
\usepackage{amsmath}
\usepackage{amssymb}
\usepackage{icomma}
%\usepackage[dvips]{graphicx}
%\usepackage{floatflt}
%\usepackage{enumitem}
%\usepackage{babel}
\usepackage{blindtext}
%\usepackage{showframe}
\usepackage{calc}
\usepackage{wrapfig}
\def\BILD{\rule{0.4\textwidth}{4cm}}

\usepackage{graphicx}
\usepackage{placeins}
\usepackage{multirow}
\usepackage{subfig}
\usepackage{url}

\renewcommand{\topfraction}{.85}
\renewcommand{\bottomfraction}{.7}
\renewcommand{\textfraction}{.15}
\renewcommand{\floatpagefraction}{.66}
\renewcommand{\dbltopfraction}{.66}
\renewcommand{\dblfloatpagefraction}{.66}
\setcounter{topnumber}{9}
\setcounter{bottomnumber}{9}
\setcounter{totalnumber}{20}
\setcounter{dbltopnumber}{9}
\setlength{\intextsep}{0cm plus1cm minus1cm}
\setlength{\parindent}{0cm}
\pdfminorversion = 5
\usepackage{setspace}
\onehalfspacing

\begin{document}
\title{Versuch 11: Atomspektren}

\date{\today}

\author{Gruppe 5a: Gia-Danh Lam, Nils Haldenwang}

\maketitle
\tableofcontents

\newpage
\section{Einleitung und theoretischer Hintergrund}
\label{sec:einleitung}

In einer Atomh�lle gebundene Elektronen k�nnen sich auf
unterschiedlichen Energieniveaus aufhalten. Durch Absorption eines
Photons der Energie $ E = h \cdot \nu $ kann das Elektron auf ein
h�heres Energieniveau angehoben werden. Dabei entspricht die Energie
des Photons genau der Energiedifferenz der beiden Niveaus. Umgekehrt
wird ein Photon entsprechender Energie emittiert, wenn ein Elektron
von einem h�heren Energieniveau auf ein niedrigeres f�llt. Da die
Energieniveaus diskret sind, m�ssen auch ihre Differenzen diskret
sein. Dies ist die Ursache daf�r, dass Emissionspektren von Atomen
stets aus einzelnen Linien bestehen. Man bezeichnet sie daher auch als
Linienspektren.

\subsection{Spektrum des Wasserstoffatoms}
\label{sec:spektr-des-wass}

F�r Wasserstoff erh�lt man als Gleichung f�r die Energie der einzelnen Schalen:

\begin{equation}
  W_{n}= R \cdot\frac{1}{n^2_1}
\label{eq:1}
\end{equation}

mit

\[R = -
\frac{e^4m}{2\hbar�}\frac{1}{(4\pi\epsilon_0)^2} =-
\frac{e^4m}{8h^2\epsilon_0^2}  \]

dabei ist $e = Elementarladung$, $m = Elektronenmasse$, $\hbar =
\frac{h}{2\pi}$ und $n = Hauptquantenzahl$. Die sog. Rydbergkonstante
hat den Wert $R = -13.605 eV$. Sie ist eine der am besten vermessenen
Naturkonstanten.

Damit erh�lt man f�r die Energiedifferenz Eines �bergangs von $n_2$ zu
$n_1$:

\begin{equation}
W_{n_2\mapsto n_1}=-R\left(\frac{1}{n_1^2}-\frac{1}{n_2^2}\right)
\label{eq:2}
\end{equation}

Mit Hilfe von $W=h\cdot \nu$, $\lambda= \frac{c}{\nu}$ und $R=
R_H\cdot h\cdot c$ ergibt sich f�r die Wellenl�nge des absorbierten/emittierten Photons:

\begin{equation}
\lambda = \frac{1}{R_H \left(\frac{1}{n_1^2}- \frac{1}{n_2^2}\right)} \text { mit } R_H=1,097\cdot 10^5 \frac{1}{cm}
\label{eq:3}
\end{equation}

Als \emph{Serien} bezeichnet man alle �berg�nge von beliebigen
Energieniveaus auf ein bestimmtes Zielniveau, also $n_1 = const.$ und
$n_2 \geq n_1 + 1$.

\begin{figure}[htb!]
  \centering
    \begin{minipage}[c]{7.2 cm}
    \includegraphics[width=8cm]{pics/dummypic}
    \caption{Spektralserien des Wasserstoffs ( nicht alle Linien
      dargestellt ).}
    \label{fig:serien}
  \end{minipage}
  \begin{minipage}[c]{7.2 cm}
    \includegraphics[width=8cm]{pics/dummypic}
    \caption{Grotrian-Diagramm f�r Kalium}
    \label{fig:grotrian}
  \end{minipage}
\end{figure}

\end{document}



