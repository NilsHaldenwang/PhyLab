\documentclass{scrartcl}
\usepackage[latin1]{inputenc}
\usepackage[T1]{fontenc}
\usepackage[ngerman]{babel}
\usepackage{amsmath}
\usepackage{amssymb}
\usepackage{icomma}
%\usepackage[dvips]{graphicx}
%\usepackage{floatflt}
%\usepackage{enumitem}
%\usepackage{babel}
\usepackage{blindtext}
%\usepackage{showframe}
\usepackage{calc}
\usepackage{wrapfig}
\def\BILD{\rule{0.4\textwidth}{4cm}}

\usepackage{graphicx}
\usepackage{placeins}
\usepackage{multirow}
\usepackage{subfig}
\usepackage{url}

\renewcommand{\topfraction}{.85}
\renewcommand{\bottomfraction}{.7}
\renewcommand{\textfraction}{.15}
\renewcommand{\floatpagefraction}{.66}
\renewcommand{\dbltopfraction}{.66}
\renewcommand{\dblfloatpagefraction}{.66}
\setcounter{topnumber}{9}
\setcounter{bottomnumber}{9}
\setcounter{totalnumber}{20}
\setcounter{dbltopnumber}{9}
\setlength{\intextsep}{0cm plus1cm minus1cm}
\pdfminorversion = 5
\usepackage{setspace}
\onehalfspacing

\begin{document}
\title{Versuch 4: Bestimmung der spezifischen Ladung $\frac{e}{m}$ des
Elektrons}

\date{15.11.2010}

\author{Gruppe 5a: Gia-Danh Lam, Nils Haldenwang}

\maketitle
\tableofcontents

\newpage

\section{Einleitung}
\label{sec:einleitung}

Als \emph{spezifische Elektronenladung} bezeichnet man das Verh�ltnis
der Ladung \emph{e} eines Elektrons zu dessen Masse \emph{m}
$\frac{e}{m}$. Die Konstante spielt insbesondere in der Atomphysik und
den angrenzenden Gebieten eine wichtige Rolle. Aus der
\emph{spezifischen Elektronenladung} l�sst sich mit Kenntnis der
\emph{Elementarladung e} die Masse eines Elektrons bestimmen.

Es existieren zahlreiche Verfahren zur Messung der \emph{spezifischen
  Elektronenladung}. Im folgenden werden zwei eher einfache
Messmethoden zur Anwendung gebracht. Zun�chst bestimmen wir die
Konstante aus der Kreisbahn von Elektronen im homogenen Magnetfeld. Im
zweiten Versuch wird ein Kr�ftegleichgewicht zwischen
Elektrostatischer Kraft und Lorentzkraft hergestellt und daraus der
gesuchte Wert berechnet.

Die Elektronenstrahlen werden in beiden Versuchen mittels einer
Gl�hkathode, einer Anode und eines Wehneltzylinders erzeugt. Der
Wehneltzylinder wurde 1902 / 1903 von Arthur Wehnelt entwickelt. Er
ist in direkter N�he der Gl�hkathode angebracht und ihr gegen�ber auf
einem negativen elektrischen Potential. Durch Einstellen dieser
Spannung l�sst sich die Intensit�t des Elektronenstrahls kalibrieren,
da langsame Elektronen das Potential nicht mehr �berwinden k�nnen.
Weiterhin werden Elektronen auf Bahnen fern der Strahlachse durch das
negative Potential der Zylinderwand abgesto�en und so der
Elektronenstrahl st�rker geb�ndelt.

Nach Einschalten der Heizspannung $U_{h}$ und Durchlaufen der
Beschleunigungsspannung $U_{a}$ haben die Elektronen eine kinetische
Energie von $E_{kin} = e \cdot U_{a}$. Tats�chlich sollte man hier
aber auch noch die thermische Geschwindigkeit ber�cksichtigen, welche die
Elektronen beim Austritt aus dem Kathodenmaterial erhalten, sodass die
effektive Beschleunigungsspannung
\begin{equation}
  \label{eq:1}
  U_{a effektiv} = U_{a} + \frac{k_B T}{2 e}
\end{equation}
verwendet werden sollte. Dabei ist $k_B$ die Boltzmannkonstante und
$T$ die Temperatur.


\subsection{Elektronen im homogenen Magnetfeld}
\label{sec:elektr-im-homog}

\subsection{Elektronen im gekreuzten homogenen elektrischen und magnetischen Feld}
\label{sec:elektr-im-gekr}

\section{Bestimmung von $\frac{e}{m}$ aus der Kreisbahn im Magnetfeld}
\label{sec:best-von-frac}

\subsection{Versuchsaufbau}
\label{sec:versuchsaufbau}

\subsection{Durchf�hrung}
\label{sec:durchfuhrung}

\subsection{Ergebnis}
\label{sec:ergebnis}

\subsection{Systematische Fehlerquellen}
\label{sec:fehlerquellen}

\section{Messungen mit elektrischem und magnetischem Feld}
\label{sec:mess-mit-elektr}

\subsection{Versuchsaufbau}
\label{sec:versuchsaufbau-1}

\subsection{Durchf�hrung}
\label{sec:durchfuhrung-1}

\subsection{Ergebnis}
\label{sec:ergebnis-1}


\end{document}