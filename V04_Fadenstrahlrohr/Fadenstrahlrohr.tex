\documentclass{scrartcl}
\usepackage[latin1]{inputenc}
\usepackage[T1]{fontenc}
\usepackage[ngerman]{babel}
\usepackage{amsmath}
\usepackage{amssymb}
\usepackage{icomma}
%\usepackage[dvips]{graphicx}
%\usepackage{floatflt}
%\usepackage{enumitem}
%\usepackage{babel}
\usepackage{blindtext}
%\usepackage{showframe}
\usepackage{calc}
\usepackage{wrapfig}
\def\BILD{\rule{0.4\textwidth}{4cm}}

\usepackage{graphicx}
\usepackage{placeins}
\usepackage{multirow}
\usepackage{subfig}
\usepackage{url}

\renewcommand{\topfraction}{.85}
\renewcommand{\bottomfraction}{.7}
\renewcommand{\textfraction}{.15}
\renewcommand{\floatpagefraction}{.66}
\renewcommand{\dbltopfraction}{.66}
\renewcommand{\dblfloatpagefraction}{.66}
\setcounter{topnumber}{9}
\setcounter{bottomnumber}{9}
\setcounter{totalnumber}{20}
\setcounter{dbltopnumber}{9}
\setlength{\intextsep}{0cm plus1cm minus1cm}
\pdfminorversion = 5
\usepackage{setspace}
\onehalfspacing

\begin{document}
\title{Versuch 4: Bestimmung der spezifischen Ladung $\frac{e}{m}$ des
Elektrons}

\date{15.11.2010}

\author{Gruppe 5a: Gia-Danh Lam, Nils Haldenwang}

\maketitle
\tableofcontents

\section{Einleitung}
\label{sec:einleitung}

\subsection{Elektronen im homogenen Magnetfeld}
\label{sec:elektr-im-homog}

\subsection{Elektronen im gekreuzten homogenen elektrischen und magnetischen Feld}
\label{sec:elektr-im-gekr}

\section{Bestimmung von $\frac{e}{m}$ aus der Kreisbahn im Magnetfeld}
\label{sec:best-von-frac}

\subsection{Versuchsaufbau}
\label{sec:versuchsaufbau}

\subsection{Durchf�hrung}
\label{sec:durchfuhrung}

\subsection{Ergebnis}
\label{sec:ergebnis}

\subsection{Systematische Fehlerquellen}
\label{sec:fehlerquellen}

\section{Messungen mit elektrischem und magnetischem Feld}
\label{sec:mess-mit-elektr}

\subsection{Versuchsaufbau}
\label{sec:versuchsaufbau-1}

\subsection{Durchf�hrung}
\label{sec:durchfuhrung-1}

\subsection{Ergebnis}
\label{sec:ergebnis-1}


\end{document}